% Slides for talk at fosdem
%
% Thomas Neidhart <thomas.neidhart@gmail.com>
% Gilles Sadowski
% January 27, 2013

\documentclass[10pt,mathserif]{beamer}
\usetheme{Boadilla}

\setbeamertemplate{footline}
{
  \leavevmode%
  \hbox{%
    \begin{beamercolorbox}[wd=.333333\paperwidth,ht=2.25ex,dp=1ex,center]{author in head/foot}%
    \usebeamerfont{author in head/foot}\insertshortinstitute
    \end{beamercolorbox}%
    \begin{beamercolorbox}[wd=.333333\paperwidth,ht=2.25ex,dp=1ex,center]{title in head/foot}%
      \usebeamerfont{title in head/foot}\insertshorttitle
    \end{beamercolorbox}%
    \begin{beamercolorbox}[wd=.333333\paperwidth,ht=2.25ex,dp=1ex,right]{date in head/foot}%
      \usebeamerfont{date in head/foot}\insertshortdate{}\hspace*{2em}
      \insertframenumber{} / \inserttotalframenumber\hspace*{2ex}
    \end{beamercolorbox}}%
  \vskip0pt%
}

%\useoutertheme[footline=authortitle]{miniframes}

\usepackage[T1]{fontenc}
\usepackage{lmodern}
\usepackage{rotating} % for defining \schwa
\usepackage{color}
\usepackage{minted}
\usepackage{longtable}

\newcommand{\fmgood}[1]{\color{green}{#1}}
\newcommand{\fmmedium}[1]{\color{orange}{#1}}
\newcommand{\fmbad}[1]{\color{red}{#1}}

\newcommand{\newauthor}[2]{
  \parbox{0.26\textwidth}{
    \texorpdfstring
      {
        \centering
        #1 \\
        {\scriptsize{\urlstyle{same}\url{#2}\urlstyle{tt}}}
      }
      {#1}
  }
}

\definecolor{links}{HTML}{2A1B81}
\hypersetup{colorlinks,linkcolor=cyan,urlcolor=magenta}

\newcommand{\schwa}{\raisebox{1ex}{\begin{turn}{180}e\end{turn}}}

%gets rid of bottom navigation bars
%\setbeamertemplate{footline}[page number]{}

%gets rid of navigation symbols
\setbeamertemplate{navigation symbols}{}

\title{Apache Commons Math}
\subtitle{A Java library of mathematical tools}
\author{
  \newauthor{Thomas Neidhart}{tn@apache.org}
\and
  \newauthor{Gilles Sadowski}{erans@apache.org}
}
\institute[FOSDEM 2013]{FOSDEM 2013}
\date{February 02, 2013}
\begin{document}

%----------- titlepage ----------------------------------------------%

\begin{frame} %[plain]
  \titlepage
\end{frame}

%----------- slide --------------------------------------------------%
\begin{frame}
  \frametitle{Overview of the talk}

\begin{itemize}
  \item What is it?
  \item What does it provide?
  \item Some simple examples
  \item Who is using it?
  \item What to do after the presentation?
\end{itemize}

\end{frame}

%----------- slide --------------------------------------------------%
\begin{frame}
  \frametitle{What is it?}

\begin{itemize}
  \item 100\% pure, self-contained Java library
  \item mathematical methods \& algorithms
  \item covers a wide range of topics
  \item aims for clean, documented and object-oriented code
  \item well tested and maintained
  \item following the Apache way
  \item Apache 2.0 license
\end{itemize}

\end{frame}

%----------- slide --------------------------------------------------%

\begin{frame}[fragile]
  \frametitle{Commons Math in Numbers}

As of January 1, 2013

\begin{itemize}
\item 57 packages\\
  829 classes\\
  77,761 lines of code\\
\item 50,512 lines of comments\\
  99.4\% documented API\\
\item 4,537 unit tests\\
  87.6\% of code covered\\
\item 98 open issues\\
  831 resolved issues\\
\item 0 dependencies\\
\end{itemize}


\end{frame}


%----------- slide --------------------------------------------------%
\begin{frame}
  \frametitle{What topics does it cover?}

% TBD - better structure, include package names

\begin{small}
\begin{itemize}
  \item Automatic Differentiation
  \item Numerical Integration
  \item Interpolation
  \item Root finders
  \item Arbitrary precision arithmetic
  \item Probability distributions
  \item Linear Algebra
  \item Fitting (non-linear least-squares, curve fitting)
  \item Ordinary differential equations (ODE)
  \item Linear/Non-linear optimization
  \item Random number generators
  \item Computational Geometry
  \item Special functions (Beta, Gamma)
  \item Kalman filter
  \item Fast Fourier transform
  \item Genetic algorithms
  \item Statistics (correlation, regression, descriptive, inference)
\end{itemize}
\end{small}
\end{frame}

%----------- slide --------------------------------------------------%
\begin{frame}
  \frametitle{FastMath}

\begin{itemize}
  \item pure Java replacement for java.lang.Math
  \item better performance for most of the functions
  \item better accuracy than Math and StrictMath
  \item currently not always faster :-(
\end{itemize}

\end{frame}

%----------- slide --------------------------------------------------%

\begin{frame}[fragile,allowframebreaks]{FastMath benchmark}
\begin{small}
Number of calls: $10^9$

Java 1.7.0\_03 (1.7.0\_03-b21) OpenJDK 64-Bit Server VM (22.0-b10)
\end{small}

\begin{columns}[t]

\begin{column}{5cm}

\begin{tiny}
\begin{longtable}{|l|r|r|r|}
\hline
\multicolumn{1}{|c|}{\textbf{Function}} & \multicolumn{1}{c|}{\textbf{Class}} & \multicolumn{1}{c|}{\textbf{Time/call (ms)}} & \multicolumn{1}{c|}{\textbf{Ratio (\%)}}\\
\hline
log   & StrictMath & 5.22349648e-05 &  100 \\
      &       Math & 2.04540923e-05 &  39 \\
      &   FastMath & 2.42863614e-05 &  \fmbad{47} \\
\hline
log10 & StrictMath & 7.06720929e-05 &  100 \\
      &       Math & 2.05568765e-05 &  29 \\
      &   FastMath & 7.56041686e-05 &  \fmbad{107} \\
\hline
log1p & StrictMath & 3.72191413e-05 &  100 \\
      &       Math & 3.78759816e-05 &  101 \\
      &   FastMath & 7.50277567e-05 &  \fmbad{201} \\
\hline
pow   & StrictMath & 1.73481768e-04 &  100 \\
      &       Math & 1.64277768e-04 &  94 \\
      &   FastMath & 1.05758357e-04 &  \fmgood{60} \\
\hline
exp   & StrictMath & 3.47634732e-05 &  100 \\
      &       Math & 2.28322028e-05 &  65 \\
      &   FastMath & 1.72182428e-05 &  \fmgood{50} \\
\hline
sin   & StrictMath & 2.86189157e-05 &  100 \\
      &       Math & 2.72834491e-05 &  95 \\
      &   FastMath & 2.13875014e-05 &  \fmgood{75} \\
\hline
asin  & StrictMath & 3.61743365e-05 &  100 \\
      &       Math & 3.63858979e-05 &  100 \\
      &   FastMath & 6.65170752e-05 &  \fmbad{184} \\
\hline
cos   & StrictMath & 3.23569717e-05 &  100 \\
      &       Math & 3.09660350e-05 &  96 \\
      &   FastMath & 2.37797205e-05 &  \fmgood{73} \\
\hline
acos  & StrictMath & 3.74091467e-05 &  100 \\
      &       Math & 3.72805271e-05 &  99 \\
      &   FastMath & 8.99709198e-05 &  \fmbad{240} \\
\hline
tan   & StrictMath & 4.36050840e-05 &  100 \\
      &       Math & 3.94326807e-05 &  90 \\
      &   FastMath & 3.96236785e-05 &  \fmgood{91} \\
\hline
\end{longtable}
\end{tiny}

\end{column}

\begin{column}{5cm}

\begin{tiny}
\begin{longtable}{|l|r|r|r|}
\hline
\multicolumn{1}{|c|}{\textbf{Function}} & \multicolumn{1}{c|}{\textbf{Class}} & \multicolumn{1}{c|}{\textbf{Time/call (ms)}} & \multicolumn{1}{c|}{\textbf{Ratio (\%)}}\\
\hline
atan  & StrictMath & 2.38076340e-05 &  100 \\
      &       Math & 2.38346576e-05 &  100 \\
      &   FastMath & 2.78287849e-05 &  \fmbad{117} \\
\hline
atan2 & StrictMath & 6.14545612e-05 &  100 \\
      &       Math & 6.25871086e-05 &  102 \\
      &   FastMath & 6.24287698e-05 &  \fmmedium{102} \\
\hline
hypot & StrictMath & 2.51933255e-04 &  100 \\
      &       Math & 2.51892886e-04 &  100 \\
      &   FastMath & 1.78561127e-05 &  \fmgood{71} \\
\hline
cbrt  & StrictMath & 7.08035203e-05 &  100 \\
      &       Math & 7.09881088e-05 &  100 \\
      &   FastMath & 4.12985279e-05 &  \fmgood{58} \\
\hline
sqrt  & StrictMath & 6.69162232e-06 &  100 \\
      &       Math & 6.63196797e-06 &  99 \\
      &   FastMath & 6.62463595e-06 &  \fmgood{99} \\
\hline
cosh  & StrictMath & 7.51256426e-05 &  100 \\
      &       Math & 7.52944441e-05 &  100 \\
      &   FastMath & 3.91062794e-05 &  \fmgood{52} \\
\hline
sinh  & StrictMath & 7.52920027e-05 &  100 \\
      &       Math & 7.58819054e-05 &  100 \\
      &   FastMath & 5.80563672e-05 &  \fmgood{77} \\
\hline
tanh  & StrictMath & 6.94914940e-05 &  100 \\
      &       Math & 7.01421819e-05 &  100 \\
      &   FastMath & 6.10102135e-05 &  \fmgood{88} \\
\hline
expm1 & StrictMath & 4.19029705e-05 &  100 \\
      &       Math & 4.18109329e-05 &  100 \\
      &   FastMath & 3.19643446e-05 &  \fmgood{76} \\
\hline
abs   & StrictMath & 5.08597800e-06 &  100 \\
      &       Math & 5.14014393e-06 &  101 \\
      &   FastMath & 5.37928625e-06 &  \fmmedium{105} \\
\hline
\end{longtable}
\end{tiny}
\end{column}

\end{columns}

\end{frame}

%----------- slide --------------------------------------------------%
\begin{frame}[fragile]
  \frametitle{Example - Gauss Integration}

\begin{minted}[xleftmargin=10pt,fontsize=\footnotesize,linenos=true]{java}
import o.a.c.m.analysis.UnivariateFunction;
import o.a.c.m.analysis.function.Cos;
import o.a.c.m.analysis.integration.gauss.GaussIntegratorFactory;
import o.a.c.m.analysis.integration.gauss.GaussIntegrator;

// ...

GaussIntegratorFactory factory = new GaussIntegratorFactory();
UnivariateFunction cos = new Cos();
// create an Gauss integrator for the interval [0, PI/2]
GaussIntegrator integrator = factory.legendre(7, 0, 0.5 * Math.PI);
double s = integrator.integrate(cos); 
\end{minted}
\end{frame}

%----------- slide --------------------------------------------------%

% \input{example.multivariate_optimizer.tex}
 
%----------- slide --------------------------------------------------%
\begin{frame}[fragile]
  \frametitle{Example - Root finder}

\begin{minted}[xleftmargin=10pt,fontsize=\footnotesize,linenos=true]{java}
import o.a.c.m.analysis.UnivariateFunction;
import o.a.c.m.analysis.solvers.BrentSolver;
import o.a.c.m.analysis.solvers.UnivariateSolver;

// ...

UnivariateFunction f = new UnivariateFunction() {
    public double value(double x) {
        return FastMath.sin(x);
    }
}
UnivariateSolver solver = new BrentSolver();
// we know that the root is somewhere between 3 and 4 ;-)
double result = solver.solve(100, f, 3, 4);
\end{minted}
\end{frame}

%----------- slide --------------------------------------------------%

\begin{frame}[fragile]
  \frametitle{Example - Eigenvalue Decomposition}

\begin{minted}[xleftmargin=10pt,fontsize=\footnotesize,linenos=true]{java}
import o.a.c.m.linear.RealMatrix;
import o.a.c.m.linear.MatrixUtils;
import o.a.c.m.linear.EigenDecomposition;

// ...

RealMatrix matrix = 
    MatrixUtils.createRealMatrix(new double[][] {
         {  5,   10,   15 },
         { 10,   20,   30 },
         { 15,   30,   45 }
});

EigenDecomposition ed = new EigenDecomposition(matrix);
double[] eigValues = ed.getRealEigenvalues();
\end{minted}
\end{frame}

%----------- slide --------------------------------------------------%

\begin{frame}[fragile]
  \frametitle{Example - Linear optimization: Simplex method}

\begin{small}
Maximize \(2x_2 + 6x_1 + 7x_0\) \\
with constraints \\
   \(\quad x_2 + 2x_1 + x_0 <= 2\) \\ 
   \(\quad -x_2 + x_1 + x_0 <= -1\) \\
   \(\quad 2x_2 - 3x_1 + x_0 <= -1\)
\end{small}

\\~\\

\begin{minted}[xleftmargin=10pt,fontsize=\tiny,linenos=true]{java}
import o.a.c.m.optim.linear.LinearObjectiveFunction;
import o.a.c.m.optim.linear.LinearConstraint;
import o.a.c.m.optim.linear.SimplexSolver;

/// ...

LinearObjectiveFunction f = new LinearObjectiveFunction(new double[] { 2, 6, 7 }, 0);

ArrayList<LinearConstraint> constraints = new ArrayList<LinearConstraint>();
constraints.add(new LinearConstraint(new double[] { 1, 2, 1 }, Relationship.LEQ, 2));
constraints.add(new LinearConstraint(new double[] { -1, 1, 1 }, Relationship.LEQ, -1));
constraints.add(new LinearConstraint(new double[] { 2, -3, 1 }, Relationship.LEQ, -1));

SimplexSolver solver = new SimplexSolver();
PointValuePair solution = solver.optimize(new MaxIter(100), f, new LinearConstraintSet(constraints),
                                          GoalType.MAXIMIZE, new NonNegativeConstraint(false));
\end{minted}
\end{frame}

%----------- slide --------------------------------------------------%
\begin{frame}
  \frametitle{Who is using Commons Math?}

Responses from our community:

\begin{tiny}
\begin{center}
\begin{longtable}{|l|p{6cm}|p{3cm}|}
\hline
\multicolumn{1}{|c|}{\textbf{Field}} & \multicolumn{1}{c|}{\textbf{Topic}} & \multicolumn{1}{c|}{\textbf{Link}}\\
\hline
Space & Orekit - open-source space dynamics library &
\url{http://www.orekit.org}\\
\hline
Astronomy & Gaia Project: data processing of astrometric, photometric, and
spectroscopic data from 1 billion stars &
\url{http://www.rssd.esa.int/index.php?project=GAIA&page=DPAC_Introduction}\\
\hline
Material Science & Simulate the mechanical behaviour of composite materials &
\url{http://navier.enpc.fr/BRISARD-Sebastien}
\\
\hline
Biotech & Nuclear Magnetic Resonance (NMR) data analysis package for molecular
structure refinement & \url{http://www.onemoonscientific.com}\\
\hline
Medical Technology &  Image processing of 3D human MRI and CT scans &
\url{http://www.stjude.org} \\
\hline
Finance & Risk managment and analysis & \url{http://www.osloclearing.no/} \\
\hline
\end{longtable}
\end{center}
\end{tiny}

A maven dependency search:

\begin{itemize}
  \item Apache Hama - parallel computing framework
  \item Myrrix - A real-time recommender system
  \item Apache Mahout - scalable machine learning
  \item Redberry - Symbolic Tensor Algebra System
  \item Facebook jcommon-stats - well you know it
  \item \ldots
\end{itemize}

\end{frame}

%----------- slide --------------------------------------------------%
\begin{frame}[fragile]
  \frametitle{How to use it yourself?}

Maven:

\begin{minted}[xleftmargin=10pt,fontsize=\small,linenos=false]{xml}
<dependency>
	<groupId>org.apache.commons</groupId>
	<artifactId>commons-math3</artifactId>
	<version>3.1.1</version>
</dependency>
\end{minted}

Download:

\url{http://commons.apache.org/math/download_math.cgi}
            
\end{frame}

%----------- slide --------------------------------------------------%
\begin{frame}
  \frametitle{Links}

\begin{itemize}
  \item Project homepage: \url{http://commons.apache.org/math/}
  \item Issue tracker: \url{https://issues.apache.org/jira/browse/MATH}
  \item Mailinglists: dev@commons.apache.org \& user@commons.apache.org \\
  e-mail subject: [math]
  \item Wiki: \url{http://wiki.apache.org/commons/MathWishList}
\end{itemize}

\end{frame}

%----------- slide --------------------------------------------------%
\begin{frame}
  \frametitle{How to contribute?}

\begin{itemize}
  \item Check out the user guide \url{http://commons.apache.org/math/userguide/index.html}
  \item Ask questions on the user mailinglist
  \item Participate in discussions on the dev mailinglist
  \item Create bug reports / feature requests / improvements in JIRA
  \item Send patches (code, documentation, examples)
  \item Provide feedback - most welcome!
\end{itemize}

\end{frame}

%----------- slide --------------------------------------------------%
\begin{frame}
  \frametitle{Hot Topics}

\begin{itemize}
  \item refactoring Linear Algebra package
  \item refactoring Optimization package
  \item Exception handling - checked vs. unchecked
  \item Add more computational geometry algorithms
  \item Upgrade to more recent Java versions (fork/join framework)
  \item (real) FastMath vs. AccurateMath (integration of jodk)
  \item improve user guide / examples
  \item lots of new feature requests \ldots
\end{itemize}

\end{frame}

%----------- slide --------------------------------------------------%
\begin{frame}
  \frametitle{The End!}

\begin{itemize}
  \item Questions?
\end{itemize}

\end{frame}

\end{document}
